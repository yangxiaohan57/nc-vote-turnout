% Options for packages loaded elsewhere
\PassOptionsToPackage{unicode}{hyperref}
\PassOptionsToPackage{hyphens}{url}
%
\documentclass[
]{article}
\usepackage{lmodern}
\usepackage{amssymb,amsmath}
\usepackage{ifxetex,ifluatex}
\ifnum 0\ifxetex 1\fi\ifluatex 1\fi=0 % if pdftex
  \usepackage[T1]{fontenc}
  \usepackage[utf8]{inputenc}
  \usepackage{textcomp} % provide euro and other symbols
\else % if luatex or xetex
  \usepackage{unicode-math}
  \defaultfontfeatures{Scale=MatchLowercase}
  \defaultfontfeatures[\rmfamily]{Ligatures=TeX,Scale=1}
\fi
% Use upquote if available, for straight quotes in verbatim environments
\IfFileExists{upquote.sty}{\usepackage{upquote}}{}
\IfFileExists{microtype.sty}{% use microtype if available
  \usepackage[]{microtype}
  \UseMicrotypeSet[protrusion]{basicmath} % disable protrusion for tt fonts
}{}
\makeatletter
\@ifundefined{KOMAClassName}{% if non-KOMA class
  \IfFileExists{parskip.sty}{%
    \usepackage{parskip}
  }{% else
    \setlength{\parindent}{0pt}
    \setlength{\parskip}{6pt plus 2pt minus 1pt}}
}{% if KOMA class
  \KOMAoptions{parskip=half}}
\makeatother
\usepackage{xcolor}
\IfFileExists{xurl.sty}{\usepackage{xurl}}{} % add URL line breaks if available
\IfFileExists{bookmark.sty}{\usepackage{bookmark}}{\usepackage{hyperref}}
\hypersetup{
  pdftitle={Team Project 2: Estrogen Bioassay and Voting in NC},
  pdfauthor={Caleb ONeel, Junbo Guan, Nikhil Bhargava, Rhayoung Park, Xiaohan Yang},
  hidelinks,
  pdfcreator={LaTeX via pandoc}}
\urlstyle{same} % disable monospaced font for URLs
\usepackage[margin=1in]{geometry}
\usepackage{graphicx}
\makeatletter
\def\maxwidth{\ifdim\Gin@nat@width>\linewidth\linewidth\else\Gin@nat@width\fi}
\def\maxheight{\ifdim\Gin@nat@height>\textheight\textheight\else\Gin@nat@height\fi}
\makeatother
% Scale images if necessary, so that they will not overflow the page
% margins by default, and it is still possible to overwrite the defaults
% using explicit options in \includegraphics[width, height, ...]{}
\setkeys{Gin}{width=\maxwidth,height=\maxheight,keepaspectratio}
% Set default figure placement to htbp
\makeatletter
\def\fps@figure{htbp}
\makeatother
\setlength{\emergencystretch}{3em} % prevent overfull lines
\providecommand{\tightlist}{%
  \setlength{\itemsep}{0pt}\setlength{\parskip}{0pt}}
\setcounter{secnumdepth}{-\maxdimen} % remove section numbering
\usepackage{booktabs}
\usepackage{longtable}
\usepackage{array}
\usepackage{multirow}
\usepackage{wrapfig}
\usepackage{float}
\usepackage{colortbl}
\usepackage{pdflscape}
\usepackage{tabu}
\usepackage{threeparttable}
\usepackage{threeparttablex}
\usepackage[normalem]{ulem}
\usepackage{makecell}

\title{Team Project 2: Estrogen Bioassay and Voting in NC}
\author{Caleb ONeel, Junbo Guan, Nikhil Bhargava, Rhayoung Park, Xiaohan
Yang}
\date{10/18/20}

\begin{document}
\maketitle

\begin{center}\rule{0.5\linewidth}{0.5pt}\end{center}

\hypertarget{part-i-estrogen-bioassay}{%
\subsection{Part I: Estrogen Bioassay}\label{part-i-estrogen-bioassay}}

\hypertarget{summary}{%
\subsubsection{Summary}\label{summary}}

This study seeks to assess the effects estrogen has on the endocrine
system. The data for this study was collected across multiple
laboratories by testing estrogen agonists and antagonists on rats and
monitoring the effect it had on the weight of a rat's uterus. Our group
will seek to determine the effect estrogen agonists and antagonists have
on the weight of the uterus by using a multilevel linear regression
model to determine the statistical effect these variables have. The lab
workers conducting this study collected information on the rat's uterus
weight and body weight, whether or not the rat was given the estrogen
agonist or antagonist, the protocol that documented the type of rat and
delivery method, and what lab group the rats belonged to.

\hypertarget{introduction}{%
\subsubsection{Introduction}\label{introduction}}

Our statistical analysis will attempt to answer several key questions
about the data presented from this study. Various laboratories were
injecting rats with either an estrogen agonist which they speculated
would increase the weight of the uterus, and with an estrogen antagonist
which they believed would decrease the weight of the rats uterus. Our
group will analyze the data they collected to examine whether or not the
weight of the uterus responded to the estrogen doses the way they
expected. Additionally, we want to see if the results of this study
varied across the different laboratories the study was conducted in. If
there are certain labs that get wildly different results, we want to
make note of that. Lastly, we would like to uncover if the rats differed
in their sensitivity to detect the agonist or antagonist effects based
upon the protocol that was used. In this case, the protocol refers to
the delivery method of the agonist or antagonist, and the age of the
rat. To answer these questions, we will construct a regression model and
examine if these variables or groups are statistically significant in
the final output.

\hypertarget{exploratory-data-analysis-eda}{%
\subsubsection{Exploratory Data Analysis
(EDA)}\label{exploratory-data-analysis-eda}}

Before beginning any analysis, we checked if there were any null values
in the data. There were 4 rows containing dots in the column `uterus'.
These rows were removed since the number of rows were minimal to
significantly affect the analysis, and also there was no other
information given to support any other assumptions we could have made.

After deleting the rows, the final dataset consisted of 2,677 unique
instances. In the dataset, there were 7 columns, and the response
variable was defined as the weight of each rats' uterus. There were 4
types of protocols, 11 groups, and 19 labs. The number of observations
were roughly similar among groups (each group had approximately 250
observations). On the other hand, the number of observations were
significantly large for Protocol A (1030 observations), followed by
Protocol B (791 observations) and Protocol C (593 observations).
Protocol D had the least observations (263 observations).

The distribution of the response variable was plotted. The histogram
showed that the weight of the uterus weight is skewed left, and
therefore a log transformation was done. The transformation did not
result in a perfect normal distribution shape; it was a bimodal
distribution. Although it was not perfect, it was still much better than
the skewed version, and thus we decided to use the log transformed
response variable. (See Appendix.)

After plotting graphs with predictors on the x-axis and the log of
uterus weights on the y-axis, we decided to bin the predictor weight. As
shown in the plot, there is a notable difference in the two groups of
rats. One group of rats were immature and small, and the other were
mature and big. A new predictor `weight\_bin' was made by assigning rats
with weight smaller than 100mg to `small' and rats with weights larger
than 100mg to `big'.

\includegraphics{team-assignment-2-final_files/figure-latex/unnamed-chunk-2-1.pdf}

Another interesting finding was that log uterus differed by lab. Labs
such as Basf and Poulenc had smaller log uterus whereas Huntingd and
Citijapa had larger log uterus weights.

\includegraphics{team-assignment-2-final_files/figure-latex/unnamed-chunk-3-1.pdf}

Moreover, the effect of ZM dose to log uterus weight seemed to differ by
protocol. The slopes for ZM vs.~Log uterus facet wrapped by lab also
seemed to show little difference.

\includegraphics{team-assignment-2-final_files/figure-latex/unnamed-chunk-4-1.pdf}
\includegraphics{team-assignment-2-final_files/figure-latex/unnamed-chunk-4-2.pdf}

\hypertarget{model}{%
\subsubsection{Model}\label{model}}

We selected our final model using a combination of anova f-tests to
determine necessary fixed variables and AIC scores to compare models
across one another. To confirm our initial hypothesis from EDA above, we
varied the intercepts of our model based on lab (contstant slope) and
protocol (varied slope by EE \& ZM). Ultimately, we finalized a
hierarchical linear model of the form:
\(y_{i} = \beta_{0} + \beta_{1}EE + \beta_{2}ZM + \beta_{3}weight\_bin + \gamma_{Lab} + \gamma_{(EE+ZM|Protocol)}\),
where \(y_{i}\) is the log of the uterus size.

When examining the model, there are no clear violations of the
linearity, normality, equal variance, and independence assumptions. The
fixed variables in the model are also not correlated with one another.

~

log(uterus)

Predictors

Estimates

CI

p

(Intercept)

4.87

4.76~--~4.99

\textless0.001

EE

0.14

0.12~--~0.16

\textless0.001

ZM

-0.57

-0.85~--~-0.29

\textless0.001

weight\_bin {[}small{]}

-1.21

-1.25~--~-1.17

\textless0.001

Random Effects

σ2

0.19

τ00 lab

0.03

τ00 protocol

0.00

τ11 protocol.EE

0.00

τ11 protocol.ZM

0.08

ρ01 protocol.EE

-0.70

ρ01 protocol.ZM

-0.80

ICC

0.18

N lab

19

N protocol

4

Observations

2677

Marginal R2 / Conditional R2

0.682 / 0.741

The baseline uterus weight, on average, for the rats in the study that
are large, had no dose of estrogen agonist, had no dose of estrogen
antagonist, is 130.32 mg.

After binning weight on small (less than 100 mg) and big (greater than
or equal to 100 mg) based on our EDA, we included the variable in our
model. Compared to a big rat, small rats, on average, have uterus that
weighs 0.30 mg less.

For every additional increase in EE dosage (in mg/kg/day), the uterus
weight, on average, for rats increases by 1.15 mg holding all other
variables constant. For every additional increase in ZM dosage (in
mg/kg/day), the uterus weight, on average, for rats decreases by 0.57
mg. Therefore, we can confirm that there is an increasing dose response
trend for EE and a decreasing dose response trend for ZM holding all
other variables constant.

To determine whether the dose response varies across labs, we varied the
model intercepts by each lab. The estimated standard deviation 0.18
describes the variation of the log of uterus weights across different
labs, holding everything else constant. Although the lab-level variance
is relatively small, it's clear that the log of uterus weights does
differ by lab. From the dotplot below, ChungKor, CitFrank, Exxon, and
Sumitomo are all labs that are outliers associated with larger log
uterus weights, while KoreaPar, Hintingd, Basf, Poulenc, are all labs
that are outliers associated with smaller log uterus weights.

To determine if the protocols differ in their sensitivity to detecting
EE and ZM effects, we varied the model intercepts by each protocol (A
through D) and varied their slopes by both EE and ZM. The estimated
standard deviation of the coefficient, 0.06, describes the
across-protocol variation attributed to the random intercepts. The
estimated standard deviation of 0.02 describes the across-protocol
variation attributed to the random slope, EE dosage, while the estimated
standard deviation of 0.28 describes the across-protocol variation
attributed to the random slope, ZM dosage.

From the analysis above, we can tell that the protocol-level variance is
small and that the log of uterus weight does not change much across
different protocol measures. Additionally, we can tell that the
across-protocol variation due to the random slope, EE dosage, is small,
and therefore the protocol does not differ the sensitivity in detecting
EE effects. This is confirmed by the dot plot below. However, the
variation due to the random slope, ZM dosage, is relatively large at
0.28 compared to the random slope EE, and we can conclude that the
protocol does differ in the sensitivity in detecting ZM effects. In
particular, from the dot plot, protocol A has the largest effect on
increasing the log of uterus weight than the other protocols. Protocol B
also seems to have a decreasing effect on the log of uterus weight in
comparison with the other protocols.

The estimated standard error 0.43 describes the unexplained variance in
the log of uterus weights that isn't explained by our other variables,
holding everything else constant.

\begin{verbatim}
## $lab
\end{verbatim}

\includegraphics{team-assignment-2-final_files/figure-latex/unnamed-chunk-7-1.pdf}

\begin{verbatim}
## 
## $protocol
\end{verbatim}

\includegraphics{team-assignment-2-final_files/figure-latex/unnamed-chunk-7-2.pdf}

\hypertarget{conclusion}{%
\subsubsection{Conclusion}\label{conclusion}}

Our statistical analysis provided answers to the questions we had at the
offset of the experiment. Our final model supported the hypothesis that
an increase in EE (estrogen agonist) did correspond to a weight gain in
the rats uterus, and that an increase in ZM dosage (estrogen antagonist)
led to a decrease in the weight of the uterus. The final model also
concluded that there is variation in the uterus's weight when grouping
by laboratory. Although the variation is small, it is apparent from our
results that the laboratory where an experiment was conducted does lead
to variance in the weight of the uterus. Finally, it appears that the
protocol appears to have some effect on the sensitivity of the rat's
reaction to EE and ZM, it is not enough to be significant.

While we are confident in the results of our analysis, we do believe
there are potential limitations to this study. Although this data has a
substantial amount of aggregate, we were ultimately attempting to break
it down by different laboratories, protocols and groups that received EE
or ZM as opposed to a control group. When all these factors are taken
into consideration, the number of observations we have for each subset
of the data becomes a bit thin. Although we do not know if this factor
ultimately came into play, there is always an increased risk that
reporting or procedures vary when the different laboratories span
different countries and continents. This can sometimes lead to slightly
different methods in data collection and recording that may occur.

\hypertarget{part-ii-voting-in-nc-2016-general-elections}{%
\subsection{Part II: Voting in NC (2016 General
Elections)}\label{part-ii-voting-in-nc-2016-general-elections}}

\hypertarget{summary-1}{%
\subsubsection{Summary}\label{summary-1}}

This analysis aims at identifying demographic groups that voted out of
those who registered in the 2016 general elections (voter turnout for
demographic groups). The datasets used are voter files in North Carolina
from The North Carolina State Board of Elections (NCSBE). The method of
this analysis is multilevel logistic regression. The conclusions are:
(1) log odds ratio of voter turnout in North Carolina does not vary by
county, (2) log odds ratio for voter turnout differs between males and
females, (3) log odds ratio of voter turnout also differs between males
and females for different party affiliation.

\hypertarget{introduction-1}{%
\subsubsection{Introduction}\label{introduction-1}}

NCSBE is the agency responsible for administering the elections process
and campaign finance disclosure and compliance. They provide voter
registration and actual voter data for general elections online. This
analysis uses the voter files for the general elections in November
2016, and it focuses on identifying if voter turnout is associated with
demographics. More specifically, this analysis is interested in
exploring if age, sex, race, and party affiliation have influences on
whether a registered North Carolinian would actually vote in November
2016. We are also interested in learning if these demographic features
have interacting effects on one another. The method used for this
analysis is multilevel/hierarchical logistic regression because voters
in North Carolina are naturally grouped by county. It is reasonable to
take advantage of this natural level and explore if voter turnout
differs by county in the 2016 elections.

\hypertarget{data}{%
\subsubsection{Data}\label{data}}

The datasets for this analysis are the registered voter dataset and the
vote dataset from NCSBE. The registered voter data is a table of all
registered voters in the 2016 elections. It contains 514,848
observations and 11 variables, in which the registered voters are
aggregated by 10 variables (county, age, party, etc.). The voter data is
a table of all actual voters in 2016. It contains 734,126 observations
and 12 variables, in which voters are aggregated by 11 variables
(county, age, party, voting method, etc.). We found these two tables
have county, race, sex, party, and age in common, so we decided these
variables to be our grouping criteria, merging conditions, and
predictors.

We needed a single table to build the model, so we decided to merge the
two datasets into one table. We first aggregated these two datasets
separately, both by county, age, party, sex, and race, and then
left-joined actual voters into registered voters. We then sampled 20
counties out of this single table.

During our data wrangling process, we discovered the left-joining
generated more than 400 observations with actual voters as nulls,
because these demographic combinations appeared in the registered voter
table but not in the actual voter table. Lacking further information for
these null values, we decided to drop these observations. Then we ended
up with a final dataset containing 3745 observations and 7 variables.

\hypertarget{exploratory-data-analysis-eda-1}{%
\subsubsection{Exploratory Data Analysis
(EDA)}\label{exploratory-data-analysis-eda-1}}

Before starting the analysis, the predictors `age', `party', `race',
`sex' and `county' was considered as factor variable. For EDA purposes,
we used turnout for the response variable. Here, turnout was defined as
the number of total voters who actually voted per each group divided by
the number of registered voters per each group. The histogram for the
turnout showed somewhat normal distribution. (See Appendix)

The plot showing turnout rate vs.~party affiliation by race revealed
that there appears to be a difference in turnout rates among different
party affiliation and race. For mixed race, American Indians and the
unknown race group, the median turnout rate for the Libertarian party
was higher than the democratic party. This was not the case for black
and white voters; black libertarian and white libertarian showed
slightly lower turnout rates compared to black democrats and white
democrats.

\includegraphics{team-assignment-2-final_files/figure-latex/unnamed-chunk-8-1.pdf}

The relationship between sex and turnout rate by party also appears to
show a difference among these groups. Female libertarians tend to show
up less compared to female democrats, but male libertarians showed up
more compared to male democrats. The median turnout value for unknown
libertarians was much higher than unknown democrats.

\includegraphics{team-assignment-2-final_files/figure-latex/unnamed-chunk-9-1.pdf}
We also drew a plot to check if turnout rate varied by county. As shown
in the plot below, counties such as Ashe, Hyde and Chatham showed high
turnout rates whereas Hoke and Pasquotank showed relatively low turnout
rates.

\includegraphics{team-assignment-2-final_files/figure-latex/unnamed-chunk-10-1.pdf}

\hypertarget{model-1}{%
\subsubsection{Model}\label{model-1}}

\hypertarget{model-selection}{%
\paragraph{Model Selection}\label{model-selection}}

We selected our final model using ANOVA f-tests. All the predictors are
categorical variables, so there is no need to center or make any
transformation.

Our main goal is to identify and estimate groups that voted in the 2016
election out of those who registered. Thus, we want to keep all the
predictors in our null model. Since we are curious about if the overall
log odds of voting differs by county in the 2016 election, we introduced
varying intercepts to our model and added county as the group variable.
Furthermore, we are interested in whether the turnout rates differ
between females and males for the different party affiliations, so we
added the interaction term between sex and party to our null model. We
tried to explore every possible interaction term between the four
predictors. For each interaction term, we added it to the null model and
did the ANOVA test. It turns out that every interaction term has a small
p-value, which means every interaction term is statistically significant
to our model. Since we have such a large amount of relevant data, every
interaction term turned out to be statistically significant. Therefore,
we decided to focus on adding interaction terms that seemed to be
relevant based on our EDA and scientific significance, and included the
interaction between race and party in our final model.

~

cbind(Total\_voter\_his,(Total\_voter\_voter -Total\_voter\_his))

Predictors

Odds Ratios

CI

p

(Intercept)

1.21

1.10~--~1.33

\textless0.001

Age {[}Age 26 - 40{]}

1.28

1.27~--~1.30

\textless0.001

Age {[}Age 41 - 65{]}

2.48

2.46~--~2.51

\textless0.001

Age {[}Age Over 66{]}

2.64

2.61~--~2.68

\textless0.001

Sex {[}M{]}

0.72

0.71~--~0.73

\textless0.001

Sex {[}U{]}

0.81

0.77~--~0.85

\textless0.001

Race {[}B{]}

0.97

0.92~--~1.03

0.276

Race {[}I{]}

0.65

0.59~--~0.71

\textless0.001

Race {[}M{]}

0.87

0.80~--~0.94

\textless0.001

Race {[}O{]}

0.76

0.72~--~0.81

\textless0.001

Race {[}U{]}

1.05

0.98~--~1.12

0.150

Race {[}W{]}

1.12

1.05~--~1.18

\textless0.001

Party {[}LIB{]}

0.72

0.49~--~1.06

0.095

Party {[}REP{]}

0.71

0.65~--~0.78

\textless0.001

Party {[}UNA{]}

0.80

0.74~--~0.86

\textless0.001

Sex {[}M{]} * Party {[}LIB{]}

1.48

1.35~--~1.63

\textless0.001

Sex {[}U{]} * Party {[}LIB{]}

1.83

1.16~--~2.90

0.010

Sex {[}M{]} * Party {[}REP{]}

1.32

1.29~--~1.34

\textless0.001

Sex {[}U{]} * Party {[}REP{]}

1.45

1.32~--~1.59

\textless0.001

Sex {[}M{]} * Party {[}UNA{]}

1.21

1.19~--~1.23

\textless0.001

Sex {[}U{]} * Party {[}UNA{]}

1.11

1.04~--~1.18

0.001

Race {[}B{]} * Party {[}LIB{]}

0.75

0.50~--~1.14

0.180

Race {[}I{]} * Party {[}LIB{]}

4.05

1.24~--~13.20

0.020

Race {[}M{]} * Party {[}LIB{]}

1.22

0.72~--~2.06

0.463

Race {[}O{]} * Party {[}LIB{]}

1.09

0.70~--~1.69

0.714

Race {[}U{]} * Party {[}LIB{]}

1.33

0.83~--~2.14

0.240

Race {[}W{]} * Party {[}LIB{]}

0.89

0.60~--~1.31

0.547

Race {[}B{]} * Party {[}REP{]}

0.61

0.55~--~0.68

\textless0.001

Race {[}I{]} * Party {[}REP{]}

1.75

1.47~--~2.07

\textless0.001

Race {[}M{]} * Party {[}REP{]}

1.65

1.39~--~1.96

\textless0.001

Race {[}O{]} * Party {[}REP{]}

1.01

0.90~--~1.13

0.899

Race {[}U{]} * Party {[}REP{]}

1.77

1.57~--~2.00

\textless0.001

Race {[}W{]} * Party {[}REP{]}

1.60

1.45~--~1.76

\textless0.001

Race {[}B{]} * Party {[}UNA{]}

0.80

0.74~--~0.86

\textless0.001

Race {[}I{]} * Party {[}UNA{]}

1.12

0.98~--~1.28

0.102

Race {[}M{]} * Party {[}UNA{]}

1.11

0.99~--~1.24

0.063

Race {[}O{]} * Party {[}UNA{]}

0.99

0.91~--~1.08

0.876

Race {[}U{]} * Party {[}UNA{]}

0.94

0.86~--~1.02

0.136

Race {[}W{]} * Party {[}UNA{]}

1.04

0.97~--~1.12

0.296

Random Effects

σ2

3.29

τ00 County

0.03

ICC

0.01

N County

20

Observations

3745

Marginal R2 / Conditional R2

0.075 / 0.083

\hypertarget{model-assessment}{%
\paragraph{Model Assessment}\label{model-assessment}}

The model's binned residual plot revealed a few problems due to a trend
that follows a linear pattern where when the log odds of turnout
increases, the average residual decreases. Moreover, about 35\% of the
observations were outside the two standard error lines. Specifically,
when the turnout rate was lower than 0.65, over half of the observations
lie outside the 2-SE lines; when the turnout rate was higher than or
equal to 0.65, over 95\% of the observations lie in between the 2-SE
lines.

From the binned residual plot, we can conclude our model fits the data
better when the turnout rate is higher versus when it is lower. This is
acceptable since the goal of this analysis is not to predict turnout
rate but to identify and estimate demographic groups that voted in the
2016 election out of those who registered. One reason for what we saw
may be due to a decent amount of rows with null values being removed,
and most of those null values seem to be due to low turnout in that
demographic group and that a large part of these null values represents
0. If that holds true, then the missing data is not at random and we
bias our analysis slightly.

\includegraphics{team-assignment-2-final_files/figure-latex/unnamed-chunk-12-1.pdf}

We also compared the AIC and BIC values for our null model and final
model (include the interaction term between race and party), our final
model has lower AIC and BIC values, which means our final model fits the
data slightly better than the null model.

\hypertarget{model-interpretation}{%
\paragraph{Model Interpretation:}\label{model-interpretation}}

\hypertarget{fixed-effect}{%
\subparagraph{Fixed Effect}\label{fixed-effect}}

The intercept is 0.19, which means that a registered Asian female
democrat voter within the age range of 18 to 25, the odds of turnout is
1.21.

Holding all variables constant, male voters have 28\% lower turning out
odds than females. As the age of voters increased, the odd ration of
turnout increased significantly. Compared to voters between 18 and 25,
the odds of turnout increased by 148\% and for voters between 41 and 65,
the odds of turnout increased by 163\% for voters over age 66.

Among all races, American Indian or Alaska Native voters are the least
active. Compared to Asian voters, they have a 36\% lower odds ratio. On
the other hand, white voters have the strongest will to vote, their turn
out odds ratio is 12\% higher than Asian voters.

Democratic voters had the strongest willingness to vote. Compared to
democrats, the turnout odds ratio is 28\% lower for libertarian voters
and 29\% lower for Republican voters. For the interaction between party
and sex, we set female democrat voters as the baseline, the odds of
turning out for the polls increased by 48\% for male libertarian voters.
The odds of turning out for the polls increased by 31\% for male
republican voters.

For the interaction between race and party, Asian democrat voters were
the baseline. The odds of turnout decreased by 40\% when the voter was
black and republican. For American Indian libertarian voters, odds of
turnout increase significantly by 306\%. For white Republican voters,
odds of turnout increase by 62\%. For mixed-race republican voters, odds
of turnout increase by 65\%.

\hypertarget{random-effect}{%
\subparagraph{Random Effect}\label{random-effect}}

The county-level standard deviation is 0.17 which describes the
cross-county variation attributed to the random intercept. The counties
do vary, but not much. As shown in the plot, the counties Hoke,
Pasquotank, Gates, and Chatham appear to be outliers. Keeping other
variables constant, the odds of turnout in Hoke, Pasquotank, and Gates
is significantly lower than other counties whereas registered voters in
Chatham are much more likely to show up and vote. Although the
county-level variance is small, every county does differ from each
other, which confirms our assumption of varying intercepts by county.

\includegraphics{team-assignment-2-final_files/figure-latex/unnamed-chunk-13-1.pdf}

\hypertarget{conclusion-1}{%
\subsubsection{Conclusion}\label{conclusion-1}}

In this analysis, the associations between demographics and voter
turnout was explored. The final model confirmed that voter turnout in
North Carolina does not vary much by county. The intercept for counties
varies by a standard deviation of 0.17, which is relatively small, so we
conclude the turnout rate across counties in NC is relatively
consistent. Also, voter turnout is associated with sex: holding all
other variables constant, the log odds ratio for voter turnout for
registered male North Carolinians are 28\% less than females'.
Additionally, we also explored how the voter turnout differs between
females and males for different party affiliations. The model tells us
that the log odds ratio for voter turnout for male democrats and
republican are also less than females' in the same parties. However, for
libertarians, male's log odds ratio for turnout is 7\% larger than
female.

There is a potential limitation for this analysis. As mentioned in the
data cleaning phase, after merging the tables, 10\% of the rows had null
values. We did not have any information about these values to make
further speculation or to conduct further processing. Hence, these
observations were dropped before modeling. We believe this could cause
limitations to our analysis. These null values might be caused from
voters who moved to other counties and voted there in November 2016. It
is also possible that these values are led by voters who did not vote.
Dropping these observations with null values might cause bias to our
model, or even lead to biased conclusions.

\begin{center}\rule{0.5\linewidth}{0.5pt}\end{center}

\end{document}
